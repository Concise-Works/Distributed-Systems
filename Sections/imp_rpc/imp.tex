\newpage 
\section{Implementing RPCs with Go}
Before we can make RPC calls in Go, we need to understand Go's typing system.
\subsection{Typing in Go}

\begin{Def}[Typing in Go]

    Go is a statically typed language, meaning every variable has a fixed type at compile time. The language provides several built-in types, including:
\textbf{Basic Types:} \snippet{int}, \snippet{float64}, \snippet{string}, \snippet{bool}, and \textbf{Composite Types:} \snippet{array}, \snippet{slice}, \snippet{map}, \snippet{struct}, \snippet{interface}.
    
    A \textbf{type} in Go can also be user-defined using the \snippet{type} keyword. For example, we can define a custom structure:
    
    \begin{lstlisting}[language=Go, numbers=none]
    type Person struct {
        Name string
        Age  int
    }
    \end{lstlisting}
    
    A \snippet{struct} groups related fields together, allowing us to represent objects with multiple attributes. We can create and use instances of this struct:
    
    \begin{lstlisting}[language=Go, numbers=none]
    p := Person{Name: "Alice", Age: 25}
    fmt.Println(p.Name) // Outputs: Alice
    \end{lstlisting}
   
\end{Def}
    
\begin{Def}[Typing Functions in Go]

    Functions are explicitly typed, meaning parameters and return values must be declared:
    
    \begin{lstlisting}[language=Go, numbers=none]
    // Takes two ints and returns an int
    func Add(x int, y int) int { 
        return x + y
    }

    // We can shorthand parameters of the same type:
    func Divide(x, y int) (int, error) { // Returns an int and an error
        if y == 0 {
            return 0, fmt.Errorf("division by zero")
        }
        return x / y, nil
    }
    \end{lstlisting}
    
    \noindent
    Here, \snippet{fmt.Errorf()} generates an error message.
    
\end{Def}
    
\begin{Def}[Methods in Go: Pointers vs. Values]

    In Go, methods can be defined for both pointer receivers (\snippet{*T}) and value receivers (\snippet{T}). A method with a pointer receiver can modify the original struct, while a method with a value receiver works on a copy.
    Consider a struct representing a rectangle:
    
    \begin{lstlisting}[language=Go, numbers=none]
    package main; import "fmt"

    type rect struct {
        width, height int
    }
    
    // Pointer receiver: Can modify the original struct
    func (r *rect) area() int {
        return r.width * r.height
    }
    
    // Value receiver: Works on a copy of rect
    func (r rect) perim() int {
        return 2*r.width + 2*r.height
    }
    
    func (r *rect) setWidth(w int) {
        r.width = w
    }
    
    func (r rect) setHeight(h int) {
        r.height = h
    }
    
    func main() {
        
        r := rect{width: 10, height: 5}
        // Converts r.area() to (&r).area() automatically
        fmt.Println("area: ", r.area())  // 50
        fmt.Println("perim:", r.perim()) // 30
    
        rp := &r
        // Converts rp.perim() to (*rp).perim() automatically
        fmt.Println("perim:", rp.perim()) // 30
        fmt.Println("area: ", rp.area())  // 50
    
        r.setWidth(20)
        r.setHeight(10)
        fmt.Println("width:", r.width) // 20
        fmt.Println("height:", r.height) // 5
    }
    \end{lstlisting}
    
    \end{Def}
    
    \newpage 

    \subsection{Go's RPC Package}
    In Go, we can use the \snippet{net/rpc} package to implement server-clint RPCs.

    \begin{Def}[Setting Up an RPC Server and Client in Go]
    
    \noindent
    \textbf{Setting Up an RPC Server}:
    To create an RPC server in Go, we need to,
    \begin{itemize}
        \item Define a \textbf{service type} (a struct) that will handle remote calls.
        \item Implement \textbf{methods} that satisfy the RPC requirements.
        \item Register the service using \snippet{rpc.Register()}.
        \item Set up a network listener using \snippet{net.Listen()}.
        \item Handle requests using \snippet{rpc.HandleHTTP()} and \snippet{http.Serve()}.
    \end{itemize}
    
    \noindent
    \textbf{RPC Method Requirements}:
    Methods must satisfy these constraints or be ignored:
    \begin{itemize}
        \item The method itself and its \textbf{type} must be exported.
        \item The method must return a single value of type \snippet{error}, and have exactly \textbf{two arguments}, both of which must be \textbf{exported or built-in types}.
        \item The method's \textbf{second argument must be a pointer}.
    \end{itemize}
    
    \noindent
    E.g. Signature,
    
    \begin{lstlisting}[language=Go, numbers=none]
    func (t *T) MethodName(argType T1, replyType *T2) error
    \end{lstlisting}
    
    \noindent
    \textbf{Setting Up an RPC Client}:
    To communicate with the RPC server, a client must,
    \begin{itemize}
        \item Establish a connection using \snippet{rpc.DialHTTP()}.
        \item Call remote methods synchronously with \snippet{client.Call()}.
        \item Call remote methods asynchronously with \snippet{client.Go()}.
    \end{itemize}
    \end{Def}

    \begin{Example}[Simple Local Arithmetic RPC (Part 1)]
        Let's create a simple Arithmetic RPC. First let's setup our file structure:\\
    \noindent
    \rule {.9\textwidth}{0.5pt}
    
        \dirtree{%
        .1 arith/.
        .2 server/.
        .3 server.go.
        .2 client.go.
        }
    \noindent
        \rule {.9\textwidth}{0.5pt}
\end{Example}
        
\newpage 

\vspace{-1em}
\begin{Example}[Simple Local Arithmetic RPC (Part 2)]

    Now to setup the server:
    \begin{lstlisting}[language=Go, numbers=none]
package main; import "fmt"; import "log"; import "net"; import "net/rpc"

    type ArithService int  // Define the service type

    type Args struct {  A, B int } // Define the arguments struct

    type Quotient struct { Quo, Rem int } // Reply struct for Dividing

    // Both Multiply and Divide methods satisfy the RPC requirements
    func (t *ArithService) Multiply(args *Args, reply *int) error {
        *reply = args.A * args.B
        return nil
    }

    func (t *ArithService) Divide(args *Args, quo *Quotient) error {
        if args.B == 0 { return fmt.Errorf("divide by zero")}
        quo.Quo = args.A / args.B
        quo.Rem = args.A % args.B
        return nil
    }

    func main() {
        arith := new(ArithService)
        err := rpc.Register(arith)
        if err != nil { log.Fatal("Register Error:", err) }

        listener, err := net.Listen("tcp", ":8080")
        if err != nil { log.Fatal("RPC Start Error:", err)}

        defer listener.Close()
        log.Println("Arithmetic RPC server is running on port 8080")

        for {
            // Server waits until a connection is made
            conn, err := listener.Accept()
            if err != nil {
                log.Println("Connection Error:", err)
                continue
            }
            // Serves the connection on a new goroutine
            go rpc.ServeConn(conn)
        }
    }
    \end{lstlisting}
    \end{Example}

\newpage 

\begin{Example}[Simple Local Arithmetic RPC (Part 3)]

    After the following client code, run the server first, then client on a separate terminals:
    \begin{lstlisting}[language=Go, numbers=none]
    package main; import "fmt"; import "log"; import "net/rpc"
    
    // Define the same structures as the server
    type Args struct {
        A, B int
    }

    type Quotient struct {
        Quo, Rem int
    }

    func main() {
        // Connect to the RPC server
        client, err := rpc.Dial("tcp", "localhost:8080")
        if err != nil {
            log.Fatal("Error dialing:", err)
        }
        defer client.Close()

        // Perform a multiplication RPC call
        args := &Args{7, 8}
        var reply int
        err = client.Call("ArithService.Multiply", args, &reply)
        if err != nil {
            log.Fatal("Multiply error:", err)
        }

        // Perform a division RPC call
        var quotient Quotient
        err = client.Call("ArithService.Divide", args, &quotient)
        if err != nil {
            log.Fatal("Divide error:", err)
        }

        // Output: 7 * 8 = 56
        fmt.Printf("ArithService: %d * %d = %d\n", args.A, args.B, reply)

        // Output: 7 / 8 = 0, remainder = 7
        fmt.Printf("ArithService: %d / %d = %d, remainder = %d\n", args.A, args.B, quotient.Quo, quotient.Rem)
    }
    \end{lstlisting}
    \end{Example}
